%!TEX root = ../main.tex

\chapter{基于深度展开网络的红外小目标设计}

本章主要介绍以下排版元素:
\begin{itemize}
    \item 列表环境(包括有序、无序、定义三种列表)
    \item 插图和表格
    \item 代码环境
    \item 数学和算法环境
\end{itemize}

列表环境有三种,
类似与HTML的\texttt{<ol>},
\texttt{<ul>},\texttt{<dl>}
三个标签。
以下是一个定义列表环境:
\begin{description}
    \item[有序列表] enumerate 默认从阿拉伯数字1开始编号\footnote{如需更改请搜索“重定义列表”}
    \item[无序列表] itemize 默认圆点标记,尽量少用
    \item[定义列表] description 语义上用于一系列简短解释
\end{description}

列表可以嵌套,比如:
\begin{enumerate}
	\item 第一级列表
	\item 第一级列表
	\begin{enumerate}
		\item 第二级列表
		\item 第二级列表
        \begin{itemize}
            \item 第三级列表
            \item 第三级列表
		\end{itemize}
		\item 第二级列表
		\item 第二级列表
	\end{enumerate}
\end{enumerate}




\section{引入深度展开网络的动机和优势}


传统红外小目标检测方法大致包括两类:基于滤波与局部对比度的人工设计方法(如顶帽变换、多尺度局部对比度、局部统计等),
以及基于背景–目标分解的优化模型方法(如低秩/平滑 + 稀疏分解)。前者实现简单、计算成本低,但对复杂多变背景适应性有限;
后者在模型层面具有较好的理论依据,却依赖人工设置正则化权重和步长,迭代次数多、推理效率不高,对不同场景的泛化能力仍然
有限。

近年来,通用卷积神经网络及目标检测框架被直接迁移到红外小目标检测任务中,在一定程度上提升了检测性能。
但与可见光大目标检测相比,直接套用黑盒深度网络存在数个突出问题:

1)红外小目标几乎不具备丰富纹理与语义结构,传统深度检测网络偏重高层语义建模,易出现漏检、多检与定位不准;

2)网络完全依赖大规模标注数据进行端到端学习,而红外小目标数据集规模有限、标注成本高,模型容易过拟合,难以适应新场景;

3)复杂深度网络结构与红外成像物理过程、背景–目标先验之间缺乏显式关联,可解释性差,性能瓶颈难以从模型机理上分析和改进。

在此背景下,**深度展开网络(deep unfolding / unrolling)**为红外小目标检测提供了兼顾“模型驱动”与“数据驱动”的新范式。
其基本思想是:以现有的背景–目标分解优化模型及其迭代算法为起点,将每一次迭代映射为网络中的一层,
并把传统算法中的固定参数(如步长、阈值、正则化权重等)替换为可学习参数,
从而将有限步迭代“展开”为一个结构受物理模型约束的深度网络。与传统深度网络相比,
引入深度展开网络的优势主要体现在以下几个方面:

(1)对比通用深度网络:嵌入先验、结构可解释

深度展开网络的每一层都对应于优化算法中的具体操作,例如背景更新、目标更新或约束更新等,
其结构直接来源于背景低秩/平滑、目标稀疏等先验假设。
与完全黑盒的卷积网络相比,这种“带有物理意义的网络层”大幅提升了网络结构的可解释性,
使网络内部计算与红外成像机理和传统先验一一对应,更便于分析和调试。

(2)对比传统优化与深度网络:参数可学习,性能与效率兼顾

传统优化方法中的正则化参数与步长常依赖经验或网格搜索,
对不同场景难以做到统一最优;而通用深度网络虽可自动学习参数,却缺乏模型约束。
深度展开网络在模型框架内将这些关键参数设为可学习变量,并允许不同层具有不同的参数配置,通过端到端训练,
有限层数就能逼近甚至超越原始优化算法经过大量迭代后的解,在保持模型结构约束的同时,显著提升检测性能与收敛效率。

(3)对比纯数据驱动方法:更强的泛化能力与鲁棒性

相比完全依赖大量标注数据的通用深度网络,深度展开网络在结构上显式编码了背景–目标分解等先验,
对复杂背景抑制的能力部分由模型本身提供,从而减轻了对大规模训练样本的依赖。
在训练样本有限的情况下,仍能保持较好泛化性能;面对不同成像条件、背景类型(海天线、云层、地面杂波等)的变化,
网络可通过学习层间参数实现自适应调整,整体鲁棒性优于传统深度检测框架。

(4)对比大规模检测网络:复杂度易控,便于工程部署

与参数量庞大、结构通用的检测网络相比,深度展开网络的深度与宽度直接由基础迭代算法的步数与变量维度决定,
研究者可根据实时性与算力约束灵活设置层数和通道数,实现复杂度–性能的可控权衡。
由于网络结构紧凑、参数量相对较小,更适合在嵌入式平台和边缘设备上部署,满足红外预警与制导等场景对实时性的严格要求。

综上,将深度展开网络引入红外小目标检测任务,可以在保留优化模型理论优势和先验约束的同时,
充分利用深度学习的表达能力与参数自适应性,在与传统深度检测网络的对比中实现检测性能、
可解释性与工程可行性的综合提升。



\section{红外小目标的数学建模与迭代求解}

本节从红外小目标成像机理出发,对图像中背景、目标以及噪声/杂波进行建模,并在此基础上构建可迭代求解的优化模型。
最后推导出适合深度展开的迭代形式,为下一节的网络结构设计奠定基础。

\subsection{红外小目标成像与分解模型}
在单帧红外成像场景中,观测到的灰度图像记为$D \in \mathbb{R}^{ H \times W},$其中 $H$ 和 $W$ 分别表示图像的高度和宽度。
通常可以将图像视作由三部分叠加而成:
\begin{align}
  D = B + T + N ,\label{分解模型}
\end{align}

其中,每一项在红外小目标场景中都有明确的物理含义和统计特性:

(1) 背景项  $B$

在空空/海天场景下,天空辐射和大尺度云层形成缓慢变化的亮度背景;在海面场景中,波浪、涌浪等构成具有纹理结构的背景;
在地面场景中,地物、道路、建筑、植被等形成中低频结构。
总体而言,背景在较大尺度上呈缓慢变化,在小尺度上具有一定平滑性和冗余性,可被认为是“低维结构”或“可压缩表示”的信号。

(2) 目标项 $T$

红外小目标的尺寸往往只有几个像素甚至亚像素,经过成像系统的点扩散函数(PSF)卷积后,在图像上表现为局部能量集中的小斑点。
小目标在全局上只占据极少数像素,具有明显稀疏性;但在局部邻域内与背景形成亮度对比,破坏背景的平均平滑性,因此可以视作背景上的局部稀疏扰动。
    
(3) 噪声与杂波项 $N$:

噪声部分来自传感器读出噪声、量化误差、热噪声等,可近似为零均值高斯或混合噪声;
杂波则主要由高频背景纹理、云边缘、海面小波浪等构成,它们在局部上也可能形成小斑点,与目标在形态上相似但缺乏稳定结构。
综合来看,$n$包含随机噪声与难以用简单背景/目标模型解释的残余成分。

一个关键的观察是:对红外小目标而言,真正的难点不在于目标太“亮”,而是背景太“聪明”——
云边缘、海浪、地物纹理都会在局部区域制造高亮小斑点,这些“伪目标”在强度、面积上与真目标极其相似,
如果不在模型中明确区分“结构化背景  $B$”和“随机/细碎杂波$N$ ”,就很难设计出稳健的检测算法。


本工作的目标是在仅观测$D$的情况下,尽可能准确地恢复目标分量$T$或其位置分布,同时抑制背景与噪声杂波的影响。
通过合理的模型和算法实现:
\begin {itemize}
    \item 将大尺度、缓变的结构提取为背景$B$;
    \item 将局部稀疏、小尺度的目标成分提取为$T$;
    \item 将随机噪声与难以建模的复杂杂波尽可能压缩到残差$N$中。
\end {itemize}




\subsection{背景与目标的先验建模}


传统鲁棒主成分分析(RPCA)模型通常将背景视为低秩,将目标视为稀疏,从而得到 $$ \min_{B,T}; |B|_* + \lambda|T|_1 \quad \text{s.t.}\quad D = B + T, $$ 其中 $|\cdot|*$ 为核范数,$|\cdot|_1$ 为 $\ell_1$ 范数。该模型在简单背景场景中具有良好的分解能力。

然而,在复杂红外场景中——
背景不仅可能近似低秩,还可能具有多子空间结构、纹理平滑性以及局部重复模式等多种特征;
目标除了稀疏性外,还具有局部对比度增强、近似点扩散形状、与周围背景显著不同的亮度分布等特性。


因此,本文不将背景和目标先验限定在具体范数形式,而是抽象为更一般的函数: $ R(B)$: 背景先验约束,\qquad $S(T)$: 目标先验约束,  其中:
$R(B)$ 可以综合低秩、平滑、结构先验等,例如核范数 + 总变分(TV)、多尺度滤波器组等;
$S(T)$ 可以综合稀疏、形状、局部对比度等,例如 $\ell_1 / \ell_0$ 稀疏约束、点扩散模板约束、局部对比度增强约束等。

从直观上看,$R(B)$先验的含义是:“背景在大范围内是可压缩的(低秩),在小范围内是平滑的”。它刻画了“背景应当是低复杂度、缓变结构”的约束。
$ S(T) $ 不仅希望“非零元素少”,还希望这些非零元素在空间上满足“小而集中”的结构特征,以便与拉长的云边缘、高频纹理区分开来。
至于噪声与杂波 $N$,其统计特性通常不易通过单一范数进行全局刻画:一方面存在近似高斯的成像噪声,另一方面还包含云层、波浪、地物纹理等复杂结构。
本文不对 $N$ 单独施加强正则,而是通过后文的重构残差显式反映当前迭代对噪声/杂波的拟合程度,并在此基础上引入噪声感知机制(第 3.4 节、第 3.5 节)。



\subsection{基本优化模型}

从物理模型\ref{分解模型}出发,将噪声与杂波 $N$ 视为观测误差,可以写出如下约束形式: 
$$ \min_{B,T} ; R(B) + \lambda S(T) \quad \text{s.t.}\quad D = B + T + N, $$ 
其中 $\lambda>0$ 为平衡背景与目标先验的权重。

对于噪声与杂波项 \( N \),假设其主部分可视为零均值、高斯型或子高斯型噪声,则对应的负对数似然可用二范数来近似。为了同时考虑噪声和可能的建模误差,将约束问题转化为无约束优化形式。

在实际求解中,更常见做法是将 $N$ 合并到基于二范数的惩罚项中,从而得到更易处理的非约束优化问题:

\begin{align}
    \min_{B,T} \mathcal{L}(B,T) = R(B) + \lambda S(T) + \frac{\mu}{2}\|D - B - T\|_F^2, \label{无约束优化}
\end{align}

其中$|\cdot|_F$ 表示 Frobenius 范数,$\mu>0$ 为数据项权重。式 \ref{无约束优化} 的含义为:

\begin{itemize}
    \item 前两项 $R(B)+\lambda S(T)$正则项 通过先验对背景和目标施加约束;
    \item 最后一项 $\frac{\mu}{2}|D-B-T|_F^2$数据保真项 强制分解结果,与观测数据 $D$ 尽可能一致,将噪声/杂波视作残差。
\end{itemize}

由于 式\ref{无约束优化}同时包含背景与目标变量,直接联合优化往往较为困难。考虑到背景和目标的物理含义存在明显差异,本文采用交替优化思想:在固定其中一个变量的情况下,优化另一个变量,从而得到一系列可迭代求解的子问题。

从红外小目标的角度看,这里隐含的“独特理解”是:

我们不直接在原图上做分类,而是先问:图像中哪些结构更像“可以由低维模型解释的背景”,
哪些结构必须“被迫用少量异常点来解释”。那些被迫用稀疏异常点来解释的结构,就自然成为小目标候选。


\subsection{背景与目标的迭代求解}

为求解公式\ref{无约束优化} ,本文采用对 \( D \) 与 \( T \) 进行交替更新的思路,从而得到一个具有明确物理含义的迭代框架。
交替优化的基本思路是:给定当前的 $(D^{k-1},T^{k-1})$,先更新背景得到 $B^{k}$,再更新目标得到 $T^{k}$,接着通过$B^{k}$、$T^{k}$重建$D^{k}$作为下一轮的输入。不断迭代即可逐步逼近最优解。下面分别推导背景子问题与目标子问题的更新形式。

\noindent \textbf{背景子问题与近端算子}

在固定目标 \( T \) 时,背景更新子问题为:
$$
B^* = \arg\min_{B} R(B) + \frac{\mu}{2}\|B + T - D\|_F^2.
$$

令$X = D - T$,则上式 等价于:
\[
B^* = \arg\min_B R(B) + \frac{\mu}{2}\|B - X\|_F^2, 
\]

其形式即为背景正则 \( R(B) \) 对应的近端映射:

\[
B^* = \operatorname{prox}_\mu^R(X) = \operatorname{prox}_\mu^R(D - T). 
\]

在实际实现中,直接按定义求解 \( \operatorname{prox}_\mu^R \)(例如做奇异值阈值化 + TV 去噪)会带来较大的计算代价且难以与卷积特征无缝结合。

为此,可以用一个浅层卷积网络来近似该近端算子:
\begin{align}
    B^k = \mathcal{H}^{(k)}(D^{k-1} - T^{k-1}), \label{eq:4.4}
\end{align}

其中:上标 \( k \) 表示第 \( k \) 次迭代/网络第 \( k \) 层;
\( \mathcal{H}^{(k)}(\cdot) \) 是具有残差结构的 CNN 模块,用来模拟“在目标抠除后的图像上提取平滑、低秩背景”的过程。



这一形式清晰地表明:背景更新的本质是对“目标抠除图像” \( D - T \) 进行一次带先验的去噪/平滑操作。后续网络结构设计中的 背景估计模块(Background Estimation Module, BEM)就是对该近端算子的一个浅层卷积近似。



\noindent \textbf{目标子问题与梯度型更新}

在固定背景 \( B \) 时,目标更新子问题为:
\begin{align}
    T^* = \arg\min_{T} \lambda S(T) + \frac{\mu}{2}\|T + B - D\|_F^2. \label{eq:4.5}
\end{align}


如果直接采用简单的 \( \ell_1 \) 范数,公式\ref{eq:4.5} 会退化为一个软阈值更新;但红外小目标的一个特点是:\(\textbf{目标形态与背景环境密切相关}\)
(例如云层前的目标与海面上的目标外观不同),因此希望目标更新过程能够自适应地利用“当前背景估计与残差信息”。

一种自然的做法是,将 \( S(T) \) 看作一般可微函数,则其关于 $T$ 的梯度为
$ [ \frac{\partial L}{\partial T} = \lambda \nabla S(T) - \mu (D - B - T)$

采用类似梯度下降(或近似的迭代收缩)方法更新 $T$,可写成:

$$ 
T^{k} = T^{k-1} - \eta^{(k)} \Big(\lambda \nabla S(T^{k-1}) - \mu (D^{k-1} - B^{k} - T^{k-1})\Big)
$$

其中 $\eta^{(k)}>0$ 是第 $k$ 次迭代的步长。

略作整理可得

$$
T^{k} = T^{k-1} + \eta^{(k)} \mu (D^{k-1} - B^{k} - T^{k-1}) - \eta^{(k)} \lambda \nabla S(T^{k-1}).
$$

当 $T^{k-1}$ 相对较小、$\eta^{(k)}\mu$ 适当选择时,可将第一项近似理解为对“背景抠除图像” $D^{k-1} - B^{k}$ 的加权利用。
为了使形式更简洁,同时便于后续网络化实现,引入两个显式系数: $\alpha^{(k)} = \eta^{(k)} \mu,\qquad \beta^{(k)} = \eta^{(k)} \lambda, $ 
则可得到如下公式: 

\begin{align}
    T^{k} \approx T^{k-1} +\alpha^{(k)}\big(D^{k-1}-B^{k}\big) - \beta^{(k)} \nabla S(T^{k-1}).\label{目标模块}
\end{align}

公式\ref{目标模块} 可以直观理解为:
第一项 $\alpha^{(k)}(D^{k-1}-B^{k})$ 利用当前“背景抠除残差”增强疑似小目标响应;
第二项 $\beta^{(k)}\nabla S(T^{k-1})$ 则对目标施加强先验约束(如稀疏性、空间形状等),抑制伪目标和过度扩散。

在后续的深度展开网络中,我们不再显式计算 $\nabla S(T)$,而是用浅层卷积网络去近似这一“数据项 + 先验项”的综合更新。






\subsection{图像重建与数据一致性}

为了与成像模型 (3-1) 保持一致,在每次迭代后,我们显式重建图像:
\[
D^k = B^k + T^k. \label{eq:3-12}
\]

若进一步希望增强对观测图像 \( D \) 的拟合,可以引入一个轻量的重建模块 \( \mathcal{M}^{(k)}(\cdot) \),对 \( B^k + T^k \) 做细致补偿:
\[
D^k = \mathcal{M}^{(k)}\left(B^k + T^k\right). \label{eq:3-13}
\]

这一操作可以理解为在网络内部实现“数据一致性”步骤:使得 \( B^k + T^k \) 不仅满足先验约束,同时尽可能接近原始观测 \( D \),将剩余的高频噪声压缩到 \( N \) 中。


\subsection{小结}

本节从红外小目标成像的物理模型出发,将观测图像分解为背景 $B$、目标 $T$ 与噪声/杂波 $N$,在低秩 + 稀疏分解思想的基础上,
引入一般背景先验 $R(B)$ 和目标先验 $S(T)$,构建了如下非约束优化模型: $ \min_{B,T} ; R(B) + \lambda S(T) + \frac{\mu}{2},|D - B - T|_F^2. $

随后,基于交替优化思想,本节推导得到:

背景更新可视为对 (D-T) 施加背景先验的近端算子,

目标更新可写成数据项与正则项梯度的加权和

噪声和杂波的影响通过重构残差 $R^{k-1} = D^{k-1} - B^{k} - T^{k-1}$ 显式表征,并借助像素级权重函数 $C^{k} = \Phi^{(k)}(R^{k-1})$ 对目标更新进行噪声感知的门控。

以及该阶段计算得到的背景和目标进行图像重建,以提供下一轮的迭代输入

这些迭代公式在形式上与经典 RPCA/低秩+稀疏模型保持一致,同时又针对红外小目标场景引入了残差驱动的像素级噪声感知机制。下一节将在此基础上,用浅层卷积网络对上述算子进行近似,将有限次迭代展开为多阶段的深度网络结构,从而得到一个兼具物理可解释性与数据驱动表达能力的红外小目标检测模型。


\section{网络结构设计}

上一节将红外小目标检测建模为$\min_{B,T} R(B) + \lambda S(T) + \frac{\mu}{2}\|D - B - T\|_F^2,$
并给出了一种同时更新背景 $B$、目标 $T$ 以及显式噪声/杂波 $N$ 的迭代形式。

借鉴深度展开网络(Deep Unfolding Network)的思想,可以将这一迭代求解过程按“阶段”展开,得到一个由若干重复结构组成的可解释深度网络,每一阶段对应一次“背景–噪声–目标–重建”的更新。

与仅对 $(B,T)$ 做两分解的展开网络不同,本文显式保留了噪声/杂波项 $N$,在每个阶段中引入噪声感知置信度模块,从重构残差中学习像素级权重,对目标更新进行噪声感知门控,以更好适应复杂低信噪比红外场景。

\subsection{总体结构与阶段展开}

设输入单帧红外图像为 \( D \in \mathbb{R}^{H \times W} \),初始化
\[
D^0 = D, \quad T^0 = 0.
\]

根据上一节的推导过程,第 \( k \) 阶段的更新可抽象为
\begin{align}
\begin{cases}
B^{k} = \mathcal{H}^{(k)}\big(D^{k-1} - T^{k-1}\big),\\
R^{k-1} = D^{k-1} - B^{k} - T^{k-1},\\
C^{k} = \mathcal{Q}^{(k)}\big(R^{k-1}\big),\\
T^{k} = T^{k-1} + C^{k} \odot \Big(\alpha^{(k)}(D^{k-1}-B^{k}) - \beta^{(k)} \Delta T^{k}\Big),\\
D^{k} = \mathcal{M}^{(k)}\big(B^{k} + T^{k}\big)
\end{cases}
\label{网络结构方程组}
\end{align}


其中:
\begin{itemize}
    \item $\mathcal{H}^{(k)}$:第 $k$ 阶段的背景估计模块(Background Estimation Module, BEM),近似背景子问题的近端算子;
    \item $R^{k-1}$:重构残差,显式反映当前阶段对噪声/杂波和未解释目标的拟合程度;
    \item $\mathcal{Q}^{(k)}$:第 $k$ 阶段的噪声感知置信度模块(Noise-aware Confidence Estimation Module, NCEM),输出像素级置信度图 $C^{k}\in[0,1]$;
    \item $\Delta T^{k}$:由目标估计模块内部 CNN 计算出的“目标梯度近似”或先验约束项;
    \item $\alpha^{(k)},\beta^{(k)}>0$:第 $k$ 阶段的可学习标量系数,用于平衡“数据项”和“先验项”的贡献;
    \item $\odot$:逐元素乘法,表示用 $C^{k}$ 对更新量进行像素级门控;
    \item $\mathcal{M}^{(k)}$:图像重建模块(Image Reconstruction Module, IRM),将背景与目标重新组合为中间图像 $D^{k}$。

\end{itemize}

网络整体由 $K$ 个阶段串联构成,各阶段参数 ${\mathcal{H}^{(k)},\mathcal{Q}^{(k)},\alpha^{(k)},\beta^{(k)}}$ 不共享,以适应不同深度处特征分布的变化。

最终输出 \( T^K \) 作为目标响应图,经阈值与连通域分析可得到小目标检测结果。


\subsection{背景估计模块 BEM:从近端算子到卷积近似}
\label{subsec:bem}

考虑到红外背景整体上呈缓慢变化、近似低秩且局部平滑的统计特性,本文不再显式求解近端算子,而是将 $\operatorname{prox}^{R}_{\mu}(\cdot)$ 视为一种在空间域中实现平滑与结构重构的可学习非线性映射,并采用轻量卷积网络对其进行近似。具体地,在第 $k$ 阶段,背景估计模块(Background Estimation Module, BEM)定义为
\[
    B^{k} = \mathcal{H}^{(k)}\big(D^{k-1} - T^{k-1}\big),
\]
其中 $\mathcal{H}^{(k)}(\cdot)$ 为第 $k$ 阶段的可学习映射。

\noindent{通道扩张的多层残差结构}

在第 $k$ 阶段,BEM 以当前重建图像与上一阶段目标估计的差作为输入,即
\[
    X^{k}_{\text{bg}} = D^{k-1} - T^{k-1},
\]
该变量在数学上对应 $(D-T)$,物理意义为“当前帧中去除目标后,由背景与噪声主导的部分”。背景估计过程可写为
\[
    \Delta B^{k} = \mathcal{G}^{(k)}\big(X^{k}_{\text{bg}}\big), \qquad
    B^{k} = X^{k}_{\text{bg}} + \Delta B^{k},
\]
其中 $\mathcal{G}^{(k)}(\cdot)$ 由多层卷积、批归一化和非线性激活构成,用于产生对背景的近端修正量 $\Delta B^{k}$,残差形式保证了数值更新的稳定性。

在具体结构上,BEM 首先通过一层 $3\times3$ 卷积将单通道输入映射到 $C_b$ 个通道:
\[
    X^{k}_{1} = \sigma\big(\operatorname{BN}(\operatorname{Conv}^{3\times3}_{1\rightarrow C_b}(X^{k}_{\text{bg}}))\big),
\]
其中 $\operatorname{Conv}^{3\times3}_{1\rightarrow C_b}$ 表示输入通道数为 $1$、输出通道数为 $C_b$ 的 $3\times3$ 卷积,$\operatorname{BN}$ 为批归一化,$\sigma(\cdot)$ 为 ReLU 激活。该通道扩张仅在首层进行一次,其作用是将单通道红外图像嵌入到一个维度为 $C_b$ 的特征空间,在这一高维空间中更充分地表征背景的低秩结构与局部纹理。

在此基础上,网络串联 $L$ 层保持通道数不变的 $3\times3$ 卷积块:
\[
    X^{k}_{\ell+1} = \sigma\big(\operatorname{BN}(\operatorname{Conv}^{3\times3}_{C_b\rightarrow C_b}(X^{k}_{\ell}))\big),
    \quad \ell = 1,\dots,L,
\]
即在固定的 $C_b$ 维特征空间内逐层细化背景表示。该部分一方面能够描述背景的整体变化趋势,另一方面也能在局部范围刻画复杂纹理与噪声形态。最后一层 $3\times3$ 卷积将特征从 $C_b$ 个通道压缩回单通道:
\[
    \Delta B^{k} = \operatorname{Conv}^{3\times3}_{C_b\rightarrow 1}(X^{k}_{L+1}),
\]
并通过
\[
    B^{k} = X^{k}_{\text{bg}} + \Delta B^{k}
\]
得到本阶段的背景估计。其中 $X^{k}_{\text{bg}}$ 提供了与观测数据一致的初始背景近似,$\Delta B^{k}$ 在此基础上进行局部结构的细致修正,残差连接有效限制了更新幅度,有利于迭代过程的收敛与稳定。

\subsubsection{与近端算子的对应关系}

从优化视角看,BEM 对应于近端算子 $\operatorname{prox}^{R}_{\mu}(D - T)$ 的一种深度卷积近似。输入 $X^{k}_{\text{bg}}$ 与模型中的 $(D-T)$ 一一对应;首层通道扩张将图像映射到 $C_b$ 维特征空间,中间的多层“$C_b\rightarrow C_b$”卷积块在该空间中迭代作用,对正则项 $R(B)$ 所隐含的低秩性与局部平滑性进行可学习建模;末层通道压缩及残差相加则对应在原始解附近实施结构化的近端修正。相比显式的 SVD 低秩分解,这种“$1\rightarrow C_b\rightarrow \dots \rightarrow C_b\rightarrow 1$”的卷积实现避免了大规模矩阵分解带来的计算负担,在保持一定物理可解释性的同时,更适合单帧红外小目标检测任务中对端到端训练与实时推理的需求。





\subsection{噪声–杂波估计模块 NCEM:显式残差建模}

在第 3 章的建模中,噪声与复杂杂波被统一记为 $N$,其能量主要体现在重构残差
\[
    R^{k-1} = D^{k-1} - B^{k} - T^{k-1}
\]
中。已有展开网络常将这部分残差直接交由数据项处理,而没有进行显式建模。

本文提出的噪声–杂波估计模块(Noise--Clutter Estimation Module, NCEM)在每一阶段显式利用该残差,估计出一个像素级的目标置信度图 $C^{k}\in[0,1]$,用于控制后续目标更新的强度。其输出由下式给出:
\begin{equation}
    C^{k} = \mathcal{Q}^{(k)}(R^{k-1}) ,
    \tag{4-3}
\end{equation}
其中 $\mathcal{Q}^{(k)}$ 为一个两层卷积的小型网络,其结构为
Conv$(3\times3)$ $\rightarrow$ ReLU $\rightarrow$ Conv$(3\times3)$ $\rightarrow$ Sigmoid,输出单通道图像 $C^{k}$。其中末端的 Sigmoid 将输出限制在区间 $[0,1]$ 内,可理解为“该位置属于目标而非纯噪声/杂波的置信度”。

在端到端训练过程中,网络通过对 $T^{k}$ 的监督自动学习到对不同残差模式的响应特性:在真实小目标附近,残差中通常包含相对稳定的结构信号,NCEM 倾向于给出较大的 $C^{k}$,从而增强后续的目标更新;而在仅由噪声和细碎杂波构成的区域,残差呈现高度随机性,NCEM 会输出较小的 $C^{k}$,相应抑制该区域的更新。因此,尽管输入是残差 $R^{k-1}$,模块输出的并非简单的“噪声图”,而是与目标存在性相关的置信度图。


\subsection{噪声感知目标估计模块 TEM:像素级门控的网络实现}

第 3 章已经给出了像素级噪声感知权重与残差驱动门控的理论形式:在第 $k$ 次迭代时,首先显式构造重构残差
\[
R^{k-1} = D^{k-1} - B^{k} - T^{k-1},
\]
并通过映射算子 $\Phi^{(k)}$ 得到像素级权重图
\[
C^{k} = \Phi^{(k)}(R^{k-1}),\quad C^{k}\in[0,1],
\]
其中 $C^{k}(i,j)$ 表示在像素 $(i,j)$ 处,本次目标更新属于“可靠、小目标相关”响应的置信度。将该权重引入目标更新公式,对整体更新量进行噪声感知门控,可得
\[
T^{k} = T^{k-1} + C^{k} \odot \Big(\alpha^{(k)}(D^{k-1}-B^{k}) - \beta^{(k)} \nabla S(T^{k-1})\Big),
\]
其中 $\alpha^{(k)}$ 和 $\beta^{(k)}$ 为第 $k$ 阶段的数据项权重与正则项步长,$\odot$ 为空间位置上的逐元素乘法。可以将 $C^{k}(i,j)$ 看作对固定步长的像素级修正:当 $C^{k}(i,j)$ 接近 0 时,对应位置几乎不更新;当 $C^{k}(i,j)$ 接近 1 时,则保持原有更新强度。

在此基础上,目标估计模块 TEM 给出了上述梯度步的具体网络实现形式。首先,为了实现 $\Phi^{(k)}$,本文构建了一个噪声感知置信度估计子网 NCEM。该子网以残差图 $R^{k-1}$ 为输入,由若干层 $3\times3$ 卷积和非线性激活堆叠而成,最后通过 Sigmoid 函数将输出压缩到 $[0,1]$ 区间,从而得到
\[
C^{k} = \sigma\big(f_{\mathrm{NCEM}}^{(k)}(R^{k-1})\big).
\]
在真实目标附近,由于背景 $B^{k}$ 已拟合掉大部分背景分量,残差 $R^{k-1}$ 通常呈现稳定且局部集中的正偏,NCEM 倾向于输出较大的 $C^{k}$,从而保留较强的更新;在纯背景区域,残差主要由噪声和纹理引起,表现为高频、随机或弱结构性扰动,NCEM 则更可能输出接近 0 的权重,从而抑制更新。这一过程与鲁棒加权最小二乘中“根据残差大小调整测量权重”的思想在形式和物理意义上是一致的。

另一方面,正则梯度 $\nabla S(T^{k-1})$ 的解析形式通常较难直接获取,因此 TEM 采用一个浅层卷积分支对其进行近似。具体做法是,将上一阶段目标估计与当前背景抠除后的残差进行通道拼接,形成
\[
X^{k}_{\mathrm{tar}} = [\,T^{k-1},\; D^{k-1}-B^{k}\,],
\]
然后通过两层 $3\times3$ 卷积与非线性激活提取目标相关先验,得到
\[
\Delta T^{k} = \mathrm{Conv}_{3\times3}\big(\sigma(\mathrm{Conv}_{3\times3}(X^{k}_{\mathrm{tar}}))\big).
\]
其中中间层采用固定通道数,最后一层将通道压回 1,使 $\Delta T^{k}$ 与 $T^{k-1}$ 在尺寸和通道上完全对齐,在功能上等价于对 $\nabla S(T^{k-1})$ 的一个数据驱动近似:网络可以在局部空间邻域内对目标边缘和细节结构进行增强,同时抑制孤立噪声与伪纹理,实现对正则先验的自适应建模。

将该近似代入理论更新式,TEM 中实际采用的更新形式为
\[
T^{k} = T^{k-1} + C^{k} \odot \Big(\alpha^{(k)}(D^{k-1}-B^{k}) - \beta^{(k)} \Delta T^{k}\Big).
\]
其中 $\alpha^{(k)}$ 控制数据项对更新方向的贡献,决定在多大程度上依赖当前残差 $(D^{k-1}-B^{k})$;$\beta^{(k)}$ 控制正则分支的步长大小,避免无约束地放大由 $\Delta T^{k}$ 产生的更新,从而在抑制噪声与保留目标之间取得平衡。由于 $\alpha^{(k)}$ 和 $\beta^{(k)}$ 作为阶段级标量参数由网络端到端学习得到,并与像素级权重 $C^{k}$ 共同作用,因此 TEM 在整体上既保留了第 3 章中带权梯度步的数学结构,又通过卷积网络与门控机制显式实现了对残差分布和目标结构的建模,使得在复杂背景和低信噪比条件下的小目标估计更加稳定、鲁棒。







\subsection{图像重建模块 IRM}


图像重建模块 \( \mathcal{M}^{(k)} \) 对应于 (3-21)。为了保持结构简洁并严格遵循物理模型 \( D = B + T + N \),
本文采用**无参数的线性重构形式**:
\[
D^k = \mathcal{M}^{(k)}(B^k + T^k) = B^k + T^k. \tag{4-7}
\]

在该设计下,每一阶段结束时都显式满足
\[
D^k \approx B^k + T^k,
\]
噪声与建模误差被保留在残差
\[
R^k = D^k - B^k - T^k
\]
中供下一阶段使用。若在实际应用中发现需要更强的重建能力,也可在 \( B^k + T^k \) 上附加一层轻量 \( 3 \times 3 \) 
卷积进行微调,但本文优先采用无参数形式以强调可解释性。



\subsection{小结 }

本章在第3章优化模型与迭代形式的基础上,构建了一种基于深度展开的红外小目标检测网络。其核心特点如下:

1. **结构上严格对应迭代过程**
网络由 \( K \) 个阶段串联,每一阶段都执行一次“背景估计 - 噪声估计 - 目标更新 - 图像重建”的流程,与 (3-22) 中的迭代步骤一一对应,保证了良好的物理和优化可解释性。


2. **背景模块 BEM 的浅层近端近似**
BEM 用两层卷积 + 残差结构近似背景近端算子,既保留了背景平滑/低维先验,又避免了显式 SVD 等高复杂度操作。


3. **噪声 - 杂波模块 NCEM 的显式建模**
NCEM 将重建残差映射为像素级置信度图 \( C^k \),为目标估计提供噪声相关的先验信息,使网络能在空间上自适应不同噪声水平和杂波强度。


4. **噪声感知的目标模块 TEM**
TEM 在构造正则梯度近似 \( \Delta T^k \) 时显式引入 \( C^k \),并通过可学习标量 \( \alpha^{(k)}, \beta^{(k)} \) 平衡数据项与正则项的贡献,实现了一种简单而有效的噪声感知目标更新机制,同时避免了直接对更新量进行矩阵乘法门控带来的不稳定性。


5. **简洁的图像重建模块 IRM**
IRM 采用 \( D^k = B^k + T^k \) 的线性重构形式,保证每一阶段都满足物理分解关系,将噪声和建模误差留在残差中供后续阶段使用。


这一网络结构充分融合了红外小目标的成像机理、优化建模以及噪声统计特性,为后续章节的损失函数设计与实验验证提供了清晰、严密的理论与结构基础。



\section{实验设计}

\subsection{实验设置}

\subsubsection{数据集}

\subsubsection{评估指标}

\subsubsection{实现细节}


\subsection{消融实验}

\subsubsection{实现细节}



\subsection{与现有最先进方法的比较}


\subsection{关于失败案例和模型先验的讨论}


\subsection{可视化分析}





\section{结果呈现,相对第三章的改进要突出}

相信您在上一章的探索学习中已经基本掌握了插入图片的方法,
但可能仍存疑虑。
现在先简单介绍浮动体的概念,
以助您理解插图环境的布局规则,
最后再介绍子图的排布以应对您更高的排版需求。
% 关于绘图,本文将在后续章节讲述
关于图的绘制,本文将在\ref{how-to-plot} 继续讲述。 % 活用ref引用,让评阅老师随处移动

当一个图片或表格太大在当前页面无法继续排版时,
一种简单的解决方案,
即是新开一页排版(Word 默认模式),
前页可能留下大段空白,十分不美观。
\LaTeX 的默认解决方案是把它们“浮动”到下一页,
与此同时将后续正文文本填充到插入点后。

插图和表格在\LaTeX 排版中默认为一个浮动体,
当排版引擎试图放置一个浮动体时,它将遵循以下规则:
\begin{enumerate}
    \item 浮动体的布局大小不得超过版心\footnote{版心是指排版文字和图表的区域,一般在页面的中心。——百度百科},否则不能通过编译(Overfull Page Error)
    \item 浮动体只能向后浮动,无法向前浮动
    \item 浮动体默认按照 h $\to$ t $\to$ b $\to$ p 规则布局
    \begin{description}
        \item[h] 排布在当前位置,如果本页所剩空间不够,忽略,检查规则 t
        \item[t] 浮动到下一页顶部
        \item[b] 浮动到下一页底部(脚注之下)
        \item[p] 浮动到一个允许出现浮动体的页面
        \item[!] 忽略浮动体放置的大多数内部参数\footnote{在下也不太懂}
    \end{description}
    \item 设置 htbp 参数的顺序不会影响默认的规则顺序
\end{enumerate}
在实践中,一般选用浮动规则[htbp], [tbp], [htp], [tp] 来完成浮动体布局。
请不要使用单一参数布局,这样极有可能出现难解的浮动问题。
不适当的浮动规则参数将导致浮动对象被放进一个队列中等待布局,
如果队列中浮动对象超过 18 个,编译时报Too Many Unprocessed Floats错误。
当需要在一页中排版的图片较多时,
您可以通过\texttt{$\backslash$clearpage}命令强制在此处排版完所有浮动体
后在排版其他内容。关于清除浮动等复杂主题,此处不再展开。

一般实践中,插图尺寸不宜超过版心一半,插图也不宜过密。
另外,可以在论文内容稳定后,
通过前置插图代码,
强行“向前浮动”,保证插图和引用处的距离不至于太远。
% 破坏语义,不宜滥用

关于本模板对浮动体的设置,参看\texttt{zjuthesis.cls},
搜索关键字“浮动体”找到对应配置。
图片引用路径在\texttt{zjuthesis.cls}里定义的\texttt{graphicspath}里,
默认情况下,\texttt{$\backslash$includegraphics}命令从论文源码根目录搜索,
如果在根目录里找到文件,则不再继续往定义引用路径搜索,
当引擎无法找到您指定的图片资源时,会导致编译错误。
注意,引用的文件名包括文件后缀。

% 现在你可以随意更动此插图代码的位置来感受一下浮动体布局的规则
\begin{figure}[htbp]
	\centering
	\begin{subfigure}[b]{.45\textwidth}  % 注意此处的尺寸控制
		\centering
		\includegraphics[width = \textwidth]{xuejian.jpg}
		\caption{仙三}\label{fig:subfig-samp1}
	\end{subfigure}
	\begin{subfigure}[b]{.45\textwidth}
		\centering
		\includegraphics[width = \textwidth]{wenhui.jpg}
		\caption{仙三外}\label{fig:subfig-samp2}
	\end{subfigure}
	\begin{subfigure}[b]{.45\textwidth}
		\centering
		\includegraphics[width = \textwidth]{lingsha.jpg}
		\caption{仙四}\label{fig:subfig-samp3}
	\end{subfigure}
	\caption{仙剑白学传}\label{fig:subfig-samp}
\end{figure}

接下来描述子图的编写,
在实际论文撰写过程中,
经常需要比较几组实验数据或场景。
此时,合乎语义的做法是为不同的组设置子图,
而不是分别设图。

多个子图组成一个单独的浮动体布局,
共用一个总图题和总引用,并可以有各自单独的子图题和引用。
本模板使用subcaption 宏包处理子图排版,如\autoref{fig:subfig-samp} 所示
论文中不可像本文一般,
平白无故地出现与行文毫无关联的图例,
而且,必须有适当的文字内容对图例做出解释。
比如,比较分析从\autoref{fig:subfig-samp1} 到\autoref{fig:subfig-samp3}
仙剑系列在白学梗方面的运用变迁。\footnote{往后数代仍有类似场景 -\_-\# (顔文字書込禁止!)}

当准备插图资源时,应该尽可能保证插图清晰,背景透明。
图中文字大小应与文中接近,不小于脚注文字大小,不大于正文段落文字大小,
框线宽度不大于2px。

如果您曾关注过图片的格式,
应该知道图片在计算机中一般分为矢量图(\autoref{fig:vector})和位图(\autoref{fig:raster})两种类型。
通俗地说,矢量图通过几何属性存储图片信息,
所以能在缩放时保持图形的几何属性。
而位图按像素点存储图片信息,在缩放时必然会丢失信息。
对于学位论文里的图例,请尽量使用矢量图,
以给评阅老师或后人精确地参考和还原实验。
常用的矢量图格式有eps, pdf, svg 和 Adobe 系列的文件格式。
其中\LaTeX 格式可以直接引用eps 和 pdf 格式的图片。

\begin{figure}[htbp]
	\centering
	\begin{subfigure}[b]{.45\textwidth}
		\centering
		\includegraphics[width = \textwidth]{vector.pdf}
		\caption{矢量图}\label{fig:vector}
	\end{subfigure}
	\begin{subfigure}[b]{.45\textwidth}
		\centering
		\includegraphics[width = \textwidth]{raster.png}
		\caption{位图}\label{fig:raster}
	\end{subfigure}
	\caption{Google Logo 的矢量图和位图比较}\label{fig:vector-raster}
\end{figure}


\section{表格}
表格与插图一样,也是浮动体单位。
在\LaTeX 中,表格的编写成本比较高,
极易引发编译错误。
初期建议同学们直接复制本模板表格进行修改。
对于只有两列的表格,建议改用列表环境完成排版。
本模板使用tabu排版表格,
使用longtabu排版超长表格。
学术论文多用线条简洁的三线表,
所谓三线就是 toprule, midrule和bottomrule 。
如\autoref{tab:tabu_test_1} 是对tabu宏包的tabu表格环境测试。
\begin{table}[htbp]
	\centering
	\caption{这是一个用tabu环境的测试用的表格}\label{tab:tabu_test_1}
    \begin{tabu}{lrr} % lrr 表示 左对齐 右对齐 右对齐
    %\begin{tabu}{|l|r|r|} % 加上竖线看看

    \toprule % 软件学院论文模板规定表头必须加粗
    \textbf{行星}     & \textbf{赤道半径}km & \textbf{公转周期}d \\
    \midrule
    水星     & 2.439  & 87.9 \\
    金星     & 6.1    & 224.682 \\
    地球     & 6378.14 & 365.24 \\
    \bottomrule
    \end{tabu}%
\end{table}

\autoref{tab:tabu_test_2} 对tabu to表格的x列模式进行测试。在表格导言区中设置为X[1]X[2]X[2],表示这三列表格的列宽比值为1:2:2,总的表格宽度由tabu to环境设置,这里设置为0.6\textbackslash linewidth。相比于tabular环境,tabu环境的列宽设置方便许多。
\begin{table}[htbp]
	\centering
	\caption{tabu环境测试表格---X列模式}\label{tab:tabu_test_2}
    \begin{tabu} to 0.6\linewidth{X[1]X[2]X[2]}
    \toprule
    \textbf{行星}     & \textbf{赤道半径}km & \textbf{公转周期}d \\  % 为了表格排版的美观 表头建议加粗
    \midrule
    水星     & 2.439  & 87.9 \\
    金星     & 6.1    & 224.682 \\
    地球     & 6378.14 & 365.24 \\
    \bottomrule
    \end{tabu}%
\end{table}

如\autoref{tab:tabu_test_3}是longtabu环境测试表格。
longtabu环境不能用在table浮动体环境中。
根据GB/T 7713.1-2006规定:如果某个表需要转页接排,
在随后的各页上应重复表的编号。
编号后跟标题(可省略)和“(续)”, % 表:「我要续…… +1
置于表上方。
续表应重复表头。

特别需要注意的是,
longtabu是基于longtable宏包开发的,
所以在zjuthesis.cls文件中已经插入了longtable宏包。
longtable环境的所有功能都可以在longtabu中使用,
如\textbackslash endhead,
\textbackslash endfirsthead,
\textbackslash endfoot,
\textbackslash endlastfoot,
和\textbackslash caption等。
具体用法请参见longtable和tabu宏包的相应文档。

\begin{longtabu}{lccc}
\caption{材料弹性模量及泊松比}\label{tab:tabu_test_3}\\
\toprule
名  称   & 弹性模量E/Gpa & 切变模量G/Gpa & 泊松比$\mu$ \\
\midrule%
\endfirsthead
\caption{材料弹性模量及泊松比(续)}\\
\toprule
名  称   & 弹性模量E/Gpa & 切变模量G/Gpa & 泊松比$\mu$ \\
\midrule%
\endhead
\bottomrule%
\endfoot
镍铬钢、合金钢 & 206    & 79.38  & 0.3 \\
碳 钢    &  196~206 & 79     & 0.3 \\
铸 钢    &  172~202 &        & 0.3 \\
球墨铸铁   &  140~154 &  73~76 & 0.3 \\
灰铸铁、白口铸铁 &  113~157 & 44     &  0.23~0.27 \\
冷拔纯铜   & 127    & 48     &   \\
轧制磷青铜  & 113    & 41     &  0.32~0.35 \\
轧制纯铜   & 108    & 39     &  0.31~0.34 \\
轧制锰青铜  & 108    & 39     & 0.35 \\
铸铝青铜   & 103    & 41     & 0.3 \\
冷拔黄铜   &  89~97 &  34~36 &  0.32~0.42 \\
轧制锌    & 82     & 31     & 0.27 \\
硬铝合金   & 70     & 26     & 0.3 \\
轧制铝    & 68     &  25~26 &  0.32~0.36 \\
铅      & 17     & 7      & 0.42 \\
玻璃     & 55     & 22     & 0.25 \\
混凝土    &  14~39 &  439~15.7 &  0.1~0.18 \\
纵纹木材   &  9.8~12 & 0.5    &   \\
横纹木材   &  0.5~0.98 &  0.44~0.64 &   \\
橡胶     & 0.00784 &        & 0.47 \\
电木     &  1.96~2.94 &  0.69~2.06 &  0.35~0.38 \\
赛璐珞    &  1.71~1.89 &  0.69~0.98 & 0.4 \\
可锻铸铁   & 152    &        &  \\
拔制铝线   & 69     &        &  \\
大理石    & 55     &        &  \\
花岗石    & 48     &        &  \\
石灰石    & 41     &        &  \\
尼龙1010 & 1.07   &        &  \\
夹布酚醛塑料 &  4~8.8 &        &  \\
石棉酚醛塑料 & 1.3    &        &  \\
高压聚乙烯  &  0.15~0.25 &        &  \\
低压聚乙烯  &  0.49~0.78 &        &  \\
聚丙烯    &  1.32~1.42 &        &  \\
硬聚氯乙烯  &  3.14~3.92 &        &  \\
聚四氟乙烯  &  1.14~1.42 &        &  \\
\end{longtabu}%


\section{代码段}

原则上,论文中应尽可能少的出现工程代码。
如果您必须引用一小段代码,
可以使用\texttt{lstlisting}设置代码环境。
本模板的代码环境默认配置在\texttt{zjuthesis.cls},
您可以搜索关键字“代码”找到配置。

本模板不鼓励引用大段代码,
所以默认情况下不为代码环境开启行号功能。
观察\autoref{code:samp-code},结合前述图表设置,
试图理解代码环境的编写。

\begin{lstlisting}[language=C++,numbers=left, numberstyle=\tiny,label=code:samp-code, caption=一段Chromium的源代码]
// Start tasks to take all the threads and block them.
  const int kNumBlockTasks = static_cast<int>(kNumWorkerThreads);
  for (int i = 0; i < kNumBlockTasks; ++i) {
    EXPECT_TRUE(pool()->PostWorkerTask(
        FROM_HERE,
        base::Bind(&TestTracker::BlockTask, tracker(), i, &blocker)));
  }
  tracker()->WaitUntilTasksBlocked(kNumWorkerThreads);

  // Setup to open the floodgates from within Shutdown().
  SetWillWaitForShutdownCallback(
      base::Bind(&TestTracker::PostBlockingTaskThenUnblockThreads,
                 scoped_refptr<TestTracker>(tracker()), pool(), &blocker,
                 kNumWorkerThreads));
  pool()->Shutdown(kNumWorkerThreads + 1);

  // Ensure that the correct number of tasks actually got run.
  tracker()->WaitUntilTasksComplete(static_cast<size_t>(kNumWorkerThreads + 1));
  tracker()->ClearCompleteSequence();
\end{lstlisting}

引用一两行代码,可以直接使用\texttt{verbatim}环境完成。
注意,此环境不会采取任何主动断行策略。
\begin{verbatim}
Error: Command failed: /bin/sh -c rsync -arvq --exclude cache
--exclude .git 
\end{verbatim}

\section{数学和算法环境}

\TeX 模板引擎创立之初就是为了更美观地排版数学公式。
在理工科的学位论文中,数学符号和数学公式必不可少
\footnote{至于定理、引理和推论等纯理科环境,本模板未作任何设定,不讨论。}。
在本模板中,数学环境由amsmath和amssymb宏包支持。
(即便没有使用公式,您应该也希望看到$a_1$, $a_2$, $C_n^m$而不是a1, a2, Cnm 吧?)

简单的行内公式,
直接在源码处编写\texttt{\$...\$}内的公式即可,
不熟习\LaTeX 公式编写的同学,
可以使用可视化的公式编辑器产生\LaTeX 代码,
这里推荐使用Daum Equation Editor完成复杂公式编辑的工作。

对于单行公式,可以使用\texttt{\$\$...\$\$}创建。

你好这里是行内公式$ (\hat{B}, \hat{T})=R(B)+\lambda S(T)+\frac{\mu}{2} \operatorname{argmin}| | B+T-D| |_{F}^{2}.$

这个是单行公式,没有公式几$$Y=\sum_{k=1}^n X_k$$
有公式编号的单行公式
\begin{align}
    (\hat{B}, \hat{T})=R(B)+\lambda S(T)+\frac{\mu}{2} \operatorname{argmin}| | B+T-D| |_{F}^{2}.
\end{align}



如果需要设定交叉引用,推荐align环境创建,如\eqref{eq:samp}所示。
\begin{align}\label{eq:samp}
    f(x) & = 2(x + 1)^{2} - 1\\                  % & 用来对齐等号
		 & = 2(x^{2} + 2x +1)-1\\
		 & = 2x^{2} + 4x + 1
\end{align}

%一个矩阵
%$$\begin{bmatrix}
%1&2&3&4\\
%5&6&7&8\\
%9&10&11&12
%\end{bmatrix}$$

计算机类的学位论文
一般少不了对研究算法的描述。
本模板选用algorithmi2e宏包排版算法环境。
详细指令使用方式参见宏包使用手册
\footnote{一般有需求排布复杂算法的同学应该有一定的科研经历}。
如\autoref{algo:duplicate2}

\begin{algorithm}
\DontPrintSemicolon
\KwIn{A sequence of integers $\langle a_1, a_2, \ldots, a_n \rangle$}
\KwOut{The index of first location with the same value as in a previous location in the sequence}
$location \gets 0$\;
$i \gets 2$\;
\While{$i \leq n \land location = 0$} {
  $j \gets 1$\;
  \While{$j < i \land location = 0$} {
    % The "l" before the If makes it so it does not expand to a second line
    \lIf{$a_i = a_j$} {
      $location \gets i$\;
    }
    \lElse{
      $j \gets j + 1$\;
    }
  }
  $i \gets i + 1$\;
}
\Return{location}\;
\caption{{\sc FindDuplicate2}}
\label{algo:duplicate2}
\end{algorithm}

\section{绘图}\label{how-to-plot}

一图胜千言,经过同学们辛苦的实验积累下的数据,
相比于冗长的文字描述,
绘图呈现的信息结构将更具可读性。
使用强大的TikZ宏包,可以绘制各式各样的图例,
比如在\ref{dirtree} 的目录结构图就是使用TikZ宏包绘制完成。
通过绘图宏包得到的是矢量图,
经过缩放后仍能精确地指导打印。
遗憾的是,
由于使用TikZ宏包绘制图例的方法艰深繁杂,
非长期钻研学术者实不可速取。

% 这里删掉了一大段TikZ宏包的使用
% 太复杂了  如果不是跟老师搞学术的话真的算了

含有大量数据的统计图,
从事数据分析工作的同学可自行使用python或R语言完成绘制,
确保输出eps或pdf格式图形,使用插图环境引入即可。

对于一般的流程图,本模板推荐使用graphviz绘图工具绘制。
相对于TikZ,graphviz已经足够适合人类掌握了。
如果坚持使用可视化工具完成此类图例的绘制,
本文推荐一个在线绘图工具\texttt{https://www.draw.io}\footnote{请保证科学上网。},
该工具可以绘制流程图、UML图、实体关系图。
另外它还支持 Dropbox 同步及输出 pdf,
通过同步论文的图片引用目录,
可以最高效的完成绘图和插图的工作。

无论用何种工具完成,
时间精力成本都不会太低。
请妥善规划您的论文撰写时间,
确保顺利毕业。

\section{关于参考文献}

硕士学位论文的参考文献,
请严格按照导师和学院规定,
注重引文质量,万不可滥引。

参考文献参照国家标准《GB/T 7714-2005: 文后参考文献著录规则》
\footnote{此标准规定的学位论文引用格式并无指定需列出是“硕士学位论文”还是“博士学位论文”},
样式文件由南京大学胡海星提供。
\begin{verbatim}
http://haixing-hu.github.io/nju-thesis/
\end{verbatim}

学校规定,参考文献采用顺序编码制,
即引文处采用序号标注,参考文献表按引文序号顺序列出。
参考文献的排版需要引入同学们自己的文献数据库,
南京大学胡海星提供了一个样例数据库,见其代码仓库内\texttt{references/test.bib}。
通过各式文献管理工具(如Zotero),您可以在论文早期工作时逐渐积累文献数据库。
通过Google学术查找一篇文献时,如\autoref{fig:gscholar} 所示,点击cite,
选择BibTeX,即可得到本文献的Bib格式的各项字段。
\begin{figure}[htbp]
    \centering
    \includegraphics[width=\textwidth]{gscholar.png}
    \caption{使用Google学术查找引文的BibTeX字段}
    \label{fig:gscholar}
\end{figure}

由Google学术提供的文献类型和字段有可能不满足胡海星前辈的设定,
注意调整。
以下是常用的文献类型:
\begin{description}
    \item[期刊]          \texttt{@article}
    \item[专著]          \texttt{@book, @inbook}
    \item[译著]          \texttt{@Book, @inBook}
    \item[会议论文集]    \texttt{@proceeding, @inproceeding}
    \item[手册]          \texttt{@manual}
    \item[网页]          \texttt{@webpage, @online}
\end{description}

\begin{itemize}
    \item 比如这是一篇中文期刊\cite{lixiaodong1999}
    \item 这是几篇英文期刊\cite{christine1998, kanamori1998}
    \item 一本中文书\cite{zh-book-1}
    \item 一本中文译著\cite{anwen1988b}
    \item 一本英文书\cite{lamport1994latex, takeuti1973}
    \item 一篇中文inproceeding\cite{nonlinear1996}
    \item 中文proceeding\cite{a2-1}
    \item 英文proceeding\cite{a2-2}
    \item 中文inproceeding\cite{aczel1998}
    \item 一篇学位论文\cite{a4-1} 
    \item 其他资料:手册\cite{ipad}报纸\cite{renminribao}网页\cite{dubash2010}
\end{itemize}

在论文中设置了一个错误或丢失的引用不会引起编译错误,
引擎会在引用标记中设一个问号。
手动编译论文的顺序一般为:
\begin{verbatim}
xelatex main
bibtex main  // 生成参考文献
xelatex main
xelatex main
\end{verbatim}
而latexmk 自动化地执行了这些步骤,所以编译时间才需要20余秒之久。







\section{本章小结}

本章划分节比较多,正式行文中请尽量避免。

传播智识,单单借助文字的力量是无力的,
即使是日常博客文章,列表、插图、表格、代码都少不了。
何况是一篇用于申请硕士学位的论文呢?

一篇学位论文集长期的科研工程实践智慧于寥寥数万字。
如何合理规划论文语义和排版元素,
让即便不熟习此领域的后人能在短时间内消化,
得以继续开物前民,
是一个值得反复求索的话题。

