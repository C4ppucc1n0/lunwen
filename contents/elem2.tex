%!TEX root = ../main.tex

\chapter{基于深度展开网络的红外小目标设计}

本章主要介绍以下排版元素:
\begin{itemize}
    \item 列表环境(包括有序、无序、定义三种列表)
    \item 插图和表格
    \item 代码环境
    \item 数学和算法环境
\end{itemize}

列表环境有三种,
类似与HTML的\texttt{<ol>},
\texttt{<ul>},\texttt{<dl>}
三个标签。
以下是一个定义列表环境:
\begin{description}
    \item[有序列表] enumerate 默认从阿拉伯数字1开始编号\footnote{如需更改请搜索“重定义列表”}
    \item[无序列表] itemize 默认圆点标记,尽量少用
    \item[定义列表] description 语义上用于一系列简短解释
\end{description}

列表可以嵌套,比如:
\begin{enumerate}
	\item 第一级列表
	\item 第一级列表
	\begin{enumerate}
		\item 第二级列表
		\item 第二级列表
        \begin{itemize}
            \item 第三级列表
            \item 第三级列表
		\end{itemize}
		\item 第二级列表
		\item 第二级列表
	\end{enumerate}
\end{enumerate}




\section{引入深度展开网络的动机和优势}


传统红外小目标检测方法大致包括两类:基于滤波与局部对比度的人工设计方法(如顶帽变换、多尺度局部对比度、局部统计等),
以及基于背景–目标分解的优化模型方法(如低秩/平滑 + 稀疏分解)。前者实现简单、计算成本低,但对复杂多变背景适应性有限;
后者在模型层面具有较好的理论依据,却依赖人工设置正则化权重和步长,迭代次数多、推理效率不高,对不同场景的泛化能力仍然
有限。

近年来,通用卷积神经网络及目标检测框架被直接迁移到红外小目标检测任务中,在一定程度上提升了检测性能。
但与可见光大目标检测相比,直接套用黑盒深度网络存在数个突出问题:

1)红外小目标几乎不具备丰富纹理与语义结构,传统深度检测网络偏重高层语义建模,易出现漏检、多检与定位不准;

2)网络完全依赖大规模标注数据进行端到端学习,而红外小目标数据集规模有限、标注成本高,模型容易过拟合,难以适应新场景;

3)复杂深度网络结构与红外成像物理过程、背景–目标先验之间缺乏显式关联,可解释性差,性能瓶颈难以从模型机理上分析和改进。

在此背景下,**深度展开网络(deep unfolding / unrolling)**为红外小目标检测提供了兼顾“模型驱动”与“数据驱动”的新范式。
其基本思想是:以现有的背景–目标分解优化模型及其迭代算法为起点,将每一次迭代映射为网络中的一层,
并把传统算法中的固定参数(如步长、阈值、正则化权重等)替换为可学习参数,
从而将有限步迭代“展开”为一个结构受物理模型约束的深度网络。与传统深度网络相比,
引入深度展开网络的优势主要体现在以下几个方面:

(1)对比通用深度网络:嵌入先验、结构可解释

深度展开网络的每一层都对应于优化算法中的具体操作,例如背景更新、目标更新或约束更新等,
其结构直接来源于背景低秩/平滑、目标稀疏等先验假设。
与完全黑盒的卷积网络相比,这种“带有物理意义的网络层”大幅提升了网络结构的可解释性,
使网络内部计算与红外成像机理和传统先验一一对应,更便于分析和调试。

(2)对比传统优化与深度网络:参数可学习,性能与效率兼顾

传统优化方法中的正则化参数与步长常依赖经验或网格搜索,
对不同场景难以做到统一最优;而通用深度网络虽可自动学习参数,却缺乏模型约束。
深度展开网络在模型框架内将这些关键参数设为可学习变量,并允许不同层具有不同的参数配置,通过端到端训练,
有限层数就能逼近甚至超越原始优化算法经过大量迭代后的解,在保持模型结构约束的同时,显著提升检测性能与收敛效率。

(3)对比纯数据驱动方法:更强的泛化能力与鲁棒性

相比完全依赖大量标注数据的通用深度网络,深度展开网络在结构上显式编码了背景–目标分解等先验,
对复杂背景抑制的能力部分由模型本身提供,从而减轻了对大规模训练样本的依赖。
在训练样本有限的情况下,仍能保持较好泛化性能;面对不同成像条件、背景类型(海天线、云层、地面杂波等)的变化,
网络可通过学习层间参数实现自适应调整,整体鲁棒性优于传统深度检测框架。

(4)对比大规模检测网络:复杂度易控,便于工程部署

与参数量庞大、结构通用的检测网络相比,深度展开网络的深度与宽度直接由基础迭代算法的步数与变量维度决定,
研究者可根据实时性与算力约束灵活设置层数和通道数,实现复杂度–性能的可控权衡。
由于网络结构紧凑、参数量相对较小,更适合在嵌入式平台和边缘设备上部署,满足红外预警与制导等场景对实时性的严格要求。

综上,将深度展开网络引入红外小目标检测任务,可以在保留优化模型理论优势和先验约束的同时,
充分利用深度学习的表达能力与参数自适应性,在与传统深度检测网络的对比中实现检测性能、
可解释性与工程可行性的综合提升。



\section{红外小目标的数学建模与迭代求解}

本节从红外小目标成像机理出发,对图像中背景、目标以及噪声/杂波进行建模,并在此基础上构建可迭代求解的优化模型。
最后推导出适合深度展开的迭代形式,为下一节的网络结构设计奠定基础。

\subsection{红外小目标成像与分解模型}
在单帧红外成像场景中,观测到的灰度图像记为$D \in \mathbb{R}^{ H \times W},$其中 $H$ 和 $W$ 分别表示图像的高度和宽度。
通常可以将图像视作由三部分叠加而成:
\begin{align}
  D = B + T + N ,\label{分解模型}
\end{align}

其中,每一项在红外小目标场景中都有明确的物理含义和统计特性:

(1) 背景项  $B$

在空空/海天场景下,天空辐射和大尺度云层形成缓慢变化的亮度背景;在海面场景中,波浪、涌浪等构成具有纹理结构的背景;
在地面场景中,地物、道路、建筑、植被等形成中低频结构。
总体而言,背景在较大尺度上呈缓慢变化,在小尺度上具有一定平滑性和冗余性,可被认为是“低维结构”或“可压缩表示”的信号。

(2) 目标项 $T$

红外小目标的尺寸往往只有几个像素甚至亚像素,经过成像系统的点扩散函数(PSF)卷积后,在图像上表现为局部能量集中的小斑点。
小目标在全局上只占据极少数像素,具有明显稀疏性;但在局部邻域内与背景形成亮度对比,破坏背景的平均平滑性,因此可以视作背景上的局部稀疏扰动。
    
(3) 噪声与杂波项 $N$:

噪声部分来自传感器读出噪声、量化误差、热噪声等,可近似为零均值高斯或混合噪声;
杂波则主要由高频背景纹理、云边缘、海面小波浪等构成,它们在局部上也可能形成小斑点,与目标在形态上相似但缺乏稳定结构。
综合来看,$n$包含随机噪声与难以用简单背景/目标模型解释的残余成分。

一个关键的观察是:对红外小目标而言,真正的难点不在于目标太“亮”,而是背景太“聪明”——
云边缘、海浪、地物纹理都会在局部区域制造高亮小斑点,这些“伪目标”在强度、面积上与真目标极其相似,
如果不在模型中明确区分“结构化背景  $B$”和“随机/细碎杂波$N$ ”,就很难设计出稳健的检测算法。


本工作的目标是在仅观测$D$的情况下,尽可能准确地恢复目标分量$T$或其位置分布,同时抑制背景与噪声杂波的影响。
通过合理的模型和算法实现:
\begin {itemize}
    \item 将大尺度、缓变的结构提取为背景$B$;
    \item 将局部稀疏、小尺度的目标成分提取为$T$;
    \item 将随机噪声与难以建模的复杂杂波尽可能压缩到残差$N$中。
\end {itemize}




\subsection{背景与目标的先验建模}


传统鲁棒主成分分析(RPCA)模型通常将背景视为低秩,将目标视为稀疏,从而得到 $$ \min_{B,T}; |B|_* + \lambda|T|_1 \quad \text{s.t.}\quad D = B + T, $$ 其中 $|\cdot|*$ 为核范数,$|\cdot|_1$ 为 $\ell_1$ 范数。该模型在简单背景场景中具有良好的分解能力。

然而,在复杂红外场景中——
背景不仅可能近似低秩,还可能具有多子空间结构、纹理平滑性以及局部重复模式等多种特征;
目标除了稀疏性外,还具有局部对比度增强、近似点扩散形状、与周围背景显著不同的亮度分布等特性。


因此,本文不将背景和目标先验限定在具体范数形式,而是抽象为更一般的函数: $ R(B)$: 背景先验约束,\qquad $S(T)$: 目标先验约束,  其中:
$R(B)$ 可以综合低秩、平滑、结构先验等,例如核范数 + 总变分(TV)、多尺度滤波器组等;
$S(T)$ 可以综合稀疏、形状、局部对比度等,例如 $\ell_1 / \ell_0$ 稀疏约束、点扩散模板约束、局部对比度增强约束等。

从直观上看,$R(B)$先验的含义是:“背景在大范围内是可压缩的(低秩),在小范围内是平滑的”。它刻画了“背景应当是低复杂度、缓变结构”的约束。
$ S(T) $ 不仅希望“非零元素少”,还希望这些非零元素在空间上满足“小而集中”的结构特征,以便与拉长的云边缘、高频纹理区分开来。
至于噪声与杂波 $N$,其统计特性通常不易通过单一范数进行全局刻画:一方面存在近似高斯的成像噪声,另一方面还包含云层、波浪、地物纹理等复杂结构。
本文不对 $N$ 单独施加强正则,而是通过后文的重构残差显式反映当前迭代对噪声/杂波的拟合程度,并在此基础上引入噪声感知机制(第 3.4 节、第 3.5 节)。



\subsection{基本优化模型}

从物理模型\ref{分解模型}出发,将噪声与杂波 $N$ 视为观测误差,可以写出如下约束形式: 
$$ \min_{B,T} ; R(B) + \lambda S(T) \quad \text{s.t.}\quad D = B + T + N, $$ 
其中 $\lambda>0$ 为平衡背景与目标先验的权重。

对于噪声与杂波项 \( N \),假设其主部分可视为零均值、高斯型或子高斯型噪声,则对应的负对数似然可用二范数来近似。为了同时考虑噪声和可能的建模误差,将约束问题转化为无约束优化形式。

在实际求解中,更常见做法是将 $N$ 合并到基于二范数的惩罚项中,从而得到更易处理的非约束优化问题:

\begin{align}
    \min_{B,T} \mathcal{L}(B,T) = R(B) + \lambda S(T) + \frac{\mu}{2}\|D - B - T\|_F^2, \label{无约束优化}
\end{align}

其中$|\cdot|_F$ 表示 Frobenius 范数,$\mu>0$ 为数据项权重。式 \ref{无约束优化} 的含义为:

\begin{itemize}
    \item 前两项 $R(B)+\lambda S(T)$正则项 通过先验对背景和目标施加约束;
    \item 最后一项 $\frac{\mu}{2}|D-B-T|_F^2$数据保真项 强制分解结果,与观测数据 $D$ 尽可能一致,将噪声/杂波视作残差。
\end{itemize}

由于 式\ref{无约束优化}同时包含背景与目标变量,直接联合优化往往较为困难。考虑到背景和目标的物理含义存在明显差异,本文采用交替优化思想:在固定其中一个变量的情况下,优化另一个变量,从而得到一系列可迭代求解的子问题。

从红外小目标的角度看,这里隐含的“独特理解”是:

我们不直接在原图上做分类,而是先问:图像中哪些结构更像“可以由低维模型解释的背景”,
哪些结构必须“被迫用少量异常点来解释”。那些被迫用稀疏异常点来解释的结构,就自然成为小目标候选。


\subsection{背景与目标的迭代求解}

为求解公式\ref{无约束优化} ,本文采用对 \( D \) 与 \( T \) 进行交替更新的思路,从而得到一个具有明确物理含义的迭代框架。
交替优化的基本思路是:给定当前的 $(D^{k-1},T^{k-1})$,先更新背景得到 $B^{k}$,再更新目标得到 $T^{k}$,接着通过$B^{k}$、$T^{k}$重建$D^{k}$作为下一轮的输入。不断迭代即可逐步逼近最优解。下面分别推导背景子问题与目标子问题的更新形式。

\noindent \textbf{背景子问题与近端算子}

在固定目标 \( T \) 时,背景更新子问题为:
$$
B^* = \arg\min_{B} R(B) + \frac{\mu}{2}\|B + T - D\|_F^2.
$$

令$X = D - T$,则上式 等价于:
\[
B^* = \arg\min_B R(B) + \frac{\mu}{2}\|B - X\|_F^2, 
\]

其形式即为背景正则 \( R(B) \) 对应的近端映射:

\[
B^* = \operatorname{prox}_\mu^R(X) = \operatorname{prox}_\mu^R(D - T). 
\]

在实际实现中,直接按定义求解 \( \operatorname{prox}_\mu^R \)(例如做奇异值阈值化 + TV 去噪)会带来较大的计算代价且难以与卷积特征无缝结合。

为此,可以用一个浅层卷积网络来近似该近端算子:
\begin{align}
    B^k = \mathcal{H}^{(k)}(D^{k-1} - T^{k-1}), \label{eq:4.4}
\end{align}

其中:上标 \( k \) 表示第 \( k \) 次迭代/网络第 \( k \) 层;
\( \mathcal{H}^{(k)}(\cdot) \) 是具有残差结构的 CNN 模块,用来模拟“在目标抠除后的图像上提取平滑、低秩背景”的过程。



这一形式清晰地表明:背景更新的本质是对“目标抠除图像” \( D - T \) 进行一次带先验的去噪/平滑操作。后续网络结构设计中的 背景估计模块(Background Estimation Module, BEM)就是对该近端算子的一个浅层卷积近似。



\noindent \textbf{目标子问题与梯度型更新}

在固定背景 \( B \) 时,目标更新子问题为:
\begin{align}
    T^* = \arg\min_{T} \lambda S(T) + \frac{\mu}{2}\|T + B - D\|_F^2. \label{eq:4.5}
\end{align}


如果直接采用简单的 \( \ell_1 \) 范数,公式\ref{eq:4.5} 会退化为一个软阈值更新;但红外小目标的一个特点是:\(\textbf{目标形态与背景环境密切相关}\)
(例如云层前的目标与海面上的目标外观不同),因此希望目标更新过程能够自适应地利用“当前背景估计与残差信息”。

一种自然的做法是,将 \( S(T) \) 看作一般可微函数,则其关于 $T$ 的梯度为
$ [ \frac{\partial L}{\partial T} = \lambda \nabla S(T) - \mu (D - B - T)$

采用类似梯度下降(或近似的迭代收缩)方法更新 $T$,可写成:

$$ 
T^{k} = T^{k-1} - \eta^{(k)} \Big(\lambda \nabla S(T^{k-1}) - \mu (D^{k-1} - B^{k} - T^{k-1})\Big)
$$

其中 $\eta^{(k)}>0$ 是第 $k$ 次迭代的步长。

略作整理可得

$$
T^{k} = T^{k-1} + \eta^{(k)} \mu (D^{k-1} - B^{k} - T^{k-1}) - \eta^{(k)} \lambda \nabla S(T^{k-1}).
$$

当 $T^{k-1}$ 相对较小、$\eta^{(k)}\mu$ 适当选择时,可将第一项近似理解为对“背景抠除图像” $D^{k-1} - B^{k}$ 的加权利用。
为了使形式更简洁,同时便于后续网络化实现,引入两个显式系数: $\alpha^{(k)} = \eta^{(k)} \mu,\qquad \beta^{(k)} = \eta^{(k)} \lambda, $ 
则可得到如下公式: 

\begin{align}
    T^{k} \approx T^{k-1} +\alpha^{(k)}\big(D^{k-1}-B^{k}\big) - \beta^{(k)} \nabla S(T^{k-1}).\label{目标模块}
\end{align}

公式\ref{目标模块} 可以直观理解为:
第一项 $\alpha^{(k)}(D^{k-1}-B^{k})$ 利用当前“背景抠除残差”增强疑似小目标响应;
第二项 $\beta^{(k)}\nabla S(T^{k-1})$ 则对目标施加强先验约束(如稀疏性、空间形状等),抑制伪目标和过度扩散。

在后续的深度展开网络中,我们不再显式计算 $\nabla S(T)$,而是用浅层卷积网络去近似这一“数据项 + 先验项”的综合更新。






\subsection{图像重建与数据一致性}

为了与成像模型 (3-1) 保持一致,在每次迭代后,我们显式重建图像:
\[
D^k = B^k + T^k. \label{eq:3-12}
\]

若进一步希望增强对观测图像 \( D \) 的拟合,可以引入一个轻量的重建模块 \( \mathcal{M}^{(k)}(\cdot) \),对 \( B^k + T^k \) 做细致补偿:
\[
D^k = \mathcal{M}^{(k)}\left(B^k + T^k\right). \label{eq:3-13}
\]

这一操作可以理解为在网络内部实现“数据一致性”步骤:使得 \( B^k + T^k \) 不仅满足先验约束,同时尽可能接近原始观测 \( D \),将剩余的高频噪声压缩到 \( N \) 中。


\subsection{小结}

本节从红外小目标成像的物理模型出发,将观测图像分解为背景 $B$、目标 $T$ 与噪声/杂波 $N$,在低秩 + 稀疏分解思想的基础上,
引入一般背景先验 $R(B)$ 和目标先验 $S(T)$,构建了如下非约束优化模型: $ \min_{B,T} ; R(B) + \lambda S(T) + \frac{\mu}{2},|D - B - T|_F^2. $

随后,基于交替优化思想,本节推导得到:

背景更新可视为对 (D-T) 施加背景先验的近端算子,

目标更新可写成数据项与正则项梯度的加权和

噪声和杂波的影响通过重构残差 $R^{k-1} = D^{k-1} - B^{k} - T^{k-1}$ 显式表征,并借助像素级权重函数 $C^{k} = \Phi^{(k)}(R^{k-1})$ 对目标更新进行噪声感知的门控。

以及该阶段计算得到的背景和目标进行图像重建,以提供下一轮的迭代输入

这些迭代公式在形式上与经典 RPCA/低秩+稀疏模型保持一致,同时又针对红外小目标场景引入了残差驱动的像素级噪声感知机制。下一节将在此基础上,用浅层卷积网络对上述算子进行近似,将有限次迭代展开为多阶段的深度网络结构,从而得到一个兼具物理可解释性与数据驱动表达能力的红外小目标检测模型。


\section{网络结构设计}

上一节将红外小目标检测建模为$\min_{B,T} R(B) + \lambda S(T) + \frac{\mu}{2}\|D - B - T\|_F^2,$
并给出了一种同时更新背景 $B$、目标 $T$ 以及显式噪声/杂波 $N$ 的迭代形式。

借鉴深度展开网络(Deep Unfolding Network)的思想,可以将这一迭代求解过程按“阶段”展开,得到一个由若干重复结构组成的可解释深度网络,每一阶段对应一次“背景–噪声–目标–重建”的更新。

与仅对 $(B,T)$ 做两分解的展开网络不同,本文显式保留了噪声/杂波项 $N$,在每个阶段中引入噪声感知置信度模块,从重构残差中学习像素级权重,对目标更新进行噪声感知门控,以更好适应复杂低信噪比红外场景。

\subsection{总体结构与阶段展开}

设输入单帧红外图像为 \( D \in \mathbb{R}^{H \times W} \),初始化
\[
D^0 = D, \quad T^0 = 0.
\]

根据上一节的推导过程,第 \( k \) 阶段的更新可抽象为
\begin{align}
\begin{cases}
B^{k} = \mathcal{H}^{(k)}\big(D^{k-1} - T^{k-1}\big),\\
R^{k-1} = D^{k-1} - B^{k} - T^{k-1},\\
C^{k} = \mathcal{Q}^{(k)}\big(R^{k-1}\big),\\
T^{k} = T^{k-1} + C^{k} \odot \Big(\alpha^{(k)}(D^{k-1}-B^{k}) - \beta^{(k)} \Delta T^{k}\Big),\\
D^{k} = \mathcal{M}^{(k)}\big(B^{k} + T^{k}\big)
\end{cases}
\label{网络结构方程组}
\end{align}


其中:
\begin{itemize}
    \item $\mathcal{H}^{(k)}$:第 $k$ 阶段的背景估计模块(Background Estimation Module, BEM),近似背景子问题的近端算子;
    \item $R^{k-1}$:重构残差,显式反映当前阶段对噪声/杂波和未解释目标的拟合程度;
    \item $\mathcal{Q}^{(k)}$:第 $k$ 阶段的噪声感知置信度模块(Noise-aware Confidence Estimation Module, NCEM),输出像素级置信度图 $C^{k}\in[0,1]$;
    \item $\Delta T^{k}$:由目标估计模块内部 CNN 计算出的“目标梯度近似”或先验约束项;
    \item $\alpha^{(k)},\beta^{(k)}>0$:第 $k$ 阶段的可学习标量系数,用于平衡“数据项”和“先验项”的贡献;
    \item $\odot$:逐元素乘法,表示用 $C^{k}$ 对更新量进行像素级门控;
    \item $\mathcal{M}^{(k)}$:图像重建模块(Image Reconstruction Module, IRM),将背景与目标重新组合为中间图像 $D^{k}$。

\end{itemize}

网络整体由 $K$ 个阶段串联构成,各阶段参数 ${\mathcal{H}^{(k)},\mathcal{Q}^{(k)},\alpha^{(k)},\beta^{(k)}}$ 不共享,以适应不同深度处特征分布的变化。

最终输出 \( T^K \) 作为目标响应图,经阈值与连通域分析可得到小目标检测结果。


\subsection{背景估计模块 BEM:从近端算子到卷积近似}
\label{subsec:bem}

考虑到红外背景整体上呈缓慢变化、近似低秩且局部平滑的统计特性,本文不再显式求解近端算子,而是将 $\operatorname{prox}^{R}_{\mu}(\cdot)$ 视为一种在空间域中实现平滑与结构重构的可学习非线性映射,并采用轻量卷积网络对其进行近似。具体地,在第 $k$ 阶段,背景估计模块(Background Estimation Module, BEM)定义为
\[
    B^{k} = \mathcal{H}^{(k)}\big(D^{k-1} - T^{k-1}\big),
\]
其中 $\mathcal{H}^{(k)}(\cdot)$ 为第 $k$ 阶段的可学习映射。

\noindent{通道扩张的多层残差结构}

在第 $k$ 阶段,BEM 以当前重建图像与上一阶段目标估计的差作为输入,即
\[
    X^{k}_{\text{bg}} = D^{k-1} - T^{k-1},
\]
该变量在数学上对应 $(D-T)$,物理意义为“当前帧中去除目标后,由背景与噪声主导的部分”。背景估计过程可写为
\[
    \Delta B^{k} = \mathcal{G}^{(k)}\big(X^{k}_{\text{bg}}\big), \qquad
    B^{k} = X^{k}_{\text{bg}} + \Delta B^{k},
\]
其中 $\mathcal{G}^{(k)}(\cdot)$ 由多层卷积、批归一化和非线性激活构成,用于产生对背景的近端修正量 $\Delta B^{k}$,残差形式保证了数值更新的稳定性。

在具体结构上,BEM 首先通过一层 $3\times3$ 卷积将单通道输入映射到 $C_b$ 个通道:
\[
    X^{k}_{1} = \sigma\big(\operatorname{BN}(\operatorname{Conv}^{3\times3}_{1\rightarrow C_b}(X^{k}_{\text{bg}}))\big),
\]
其中 $\operatorname{Conv}^{3\times3}_{1\rightarrow C_b}$ 表示输入通道数为 $1$、输出通道数为 $C_b$ 的 $3\times3$ 卷积,$\operatorname{BN}$ 为批归一化,$\sigma(\cdot)$ 为 ReLU 激活。该通道扩张仅在首层进行一次,其作用是将单通道红外图像嵌入到一个维度为 $C_b$ 的特征空间,在这一高维空间中更充分地表征背景的低秩结构与局部纹理。

在此基础上,网络串联 $L$ 层保持通道数不变的 $3\times3$ 卷积块:
\[
    X^{k}_{\ell+1} = \sigma\big(\operatorname{BN}(\operatorname{Conv}^{3\times3}_{C_b\rightarrow C_b}(X^{k}_{\ell}))\big),
    \quad \ell = 1,\dots,L,
\]
即在固定的 $C_b$ 维特征空间内逐层细化背景表示。该部分一方面能够描述背景的整体变化趋势,另一方面也能在局部范围刻画复杂纹理与噪声形态。最后一层 $3\times3$ 卷积将特征从 $C_b$ 个通道压缩回单通道:
\[
    \Delta B^{k} = \operatorname{Conv}^{3\times3}_{C_b\rightarrow 1}(X^{k}_{L+1}),
\]
并通过
\[
    B^{k} = X^{k}_{\text{bg}} + \Delta B^{k}
\]
得到本阶段的背景估计。其中 $X^{k}_{\text{bg}}$ 提供了与观测数据一致的初始背景近似,$\Delta B^{k}$ 在此基础上进行局部结构的细致修正,残差连接有效限制了更新幅度,有利于迭代过程的收敛与稳定。

\subsubsection{与近端算子的对应关系}

从优化视角看,BEM 对应于近端算子 $\operatorname{prox}^{R}_{\mu}(D - T)$ 的一种深度卷积近似。输入 $X^{k}_{\text{bg}}$ 与模型中的 $(D-T)$ 一一对应;首层通道扩张将图像映射到 $C_b$ 维特征空间,中间的多层“$C_b\rightarrow C_b$”卷积块在该空间中迭代作用,对正则项 $R(B)$ 所隐含的低秩性与局部平滑性进行可学习建模;末层通道压缩及残差相加则对应在原始解附近实施结构化的近端修正。相比显式的 SVD 低秩分解,这种“$1\rightarrow C_b\rightarrow \dots \rightarrow C_b\rightarrow 1$”的卷积实现避免了大规模矩阵分解带来的计算负担,在保持一定物理可解释性的同时,更适合单帧红外小目标检测任务中对端到端训练与实时推理的需求。





\subsection{噪声–杂波估计模块 NCEM:显式残差建模}

在第 3 章的建模中,噪声与复杂杂波被统一记为 $N$,其能量主要体现在重构残差
\[
    R^{k-1} = D^{k-1} - B^{k} - T^{k-1}
\]
中。已有展开网络常将这部分残差直接交由数据项处理,而没有进行显式建模。

本文提出的噪声–杂波估计模块(Noise--Clutter Estimation Module, NCEM)在每一阶段显式利用该残差,估计出一个像素级的目标置信度图 $C^{k}\in[0,1]$,用于控制后续目标更新的强度。其输出由下式给出:
\begin{equation}
    C^{k} = \mathcal{Q}^{(k)}(R^{k-1}) ,
    \tag{4-3}
\end{equation}
其中 $\mathcal{Q}^{(k)}$ 为一个两层卷积的小型网络,其结构为
Conv$(3\times3)$ $\rightarrow$ ReLU $\rightarrow$ Conv$(3\times3)$ $\rightarrow$ Sigmoid,输出单通道图像 $C^{k}$。其中末端的 Sigmoid 将输出限制在区间 $[0,1]$ 内,可理解为“该位置属于目标而非纯噪声/杂波的置信度”。

在端到端训练过程中,网络通过对 $T^{k}$ 的监督自动学习到对不同残差模式的响应特性:在真实小目标附近,残差中通常包含相对稳定的结构信号,NCEM 倾向于给出较大的 $C^{k}$,从而增强后续的目标更新;而在仅由噪声和细碎杂波构成的区域,残差呈现高度随机性,NCEM 会输出较小的 $C^{k}$,相应抑制该区域的更新。因此,尽管输入是残差 $R^{k-1}$,模块输出的并非简单的“噪声图”,而是与目标存在性相关的置信度图。


\subsection{噪声感知目标估计模块 TEM:像素级门控的网络实现}

第 3 章已经给出了像素级噪声感知权重与残差驱动门控的理论形式:在第 $k$ 次迭代时,首先显式构造重构残差
\[
R^{k-1} = D^{k-1} - B^{k} - T^{k-1},
\]
并通过映射算子 $\Phi^{(k)}$ 得到像素级权重图
\[
C^{k} = \Phi^{(k)}(R^{k-1}),\quad C^{k}\in[0,1],
\]
其中 $C^{k}(i,j)$ 表示在像素 $(i,j)$ 处,本次目标更新属于“可靠、小目标相关”响应的置信度。将该权重引入目标更新公式,对整体更新量进行噪声感知门控,可得
\[
T^{k} = T^{k-1} + C^{k} \odot \Big(\alpha^{(k)}(D^{k-1}-B^{k}) - \beta^{(k)} \nabla S(T^{k-1})\Big),
\]
其中 $\alpha^{(k)}$ 和 $\beta^{(k)}$ 为第 $k$ 阶段的数据项权重与正则项步长,$\odot$ 为空间位置上的逐元素乘法。可以将 $C^{k}(i,j)$ 看作对固定步长的像素级修正:当 $C^{k}(i,j)$ 接近 0 时,对应位置几乎不更新;当 $C^{k}(i,j)$ 接近 1 时,则保持原有更新强度。

在此基础上,目标估计模块 TEM 给出了上述梯度步的具体网络实现形式。首先,为了实现 $\Phi^{(k)}$,本文构建了一个噪声感知置信度估计子网 NCEM。该子网以残差图 $R^{k-1}$ 为输入,由若干层 $3\times3$ 卷积和非线性激活堆叠而成,最后通过 Sigmoid 函数将输出压缩到 $[0,1]$ 区间,从而得到
\[
C^{k} = \sigma\big(f_{\mathrm{NCEM}}^{(k)}(R^{k-1})\big).
\]
在真实目标附近,由于背景 $B^{k}$ 已拟合掉大部分背景分量,残差 $R^{k-1}$ 通常呈现稳定且局部集中的正偏,NCEM 倾向于输出较大的 $C^{k}$,从而保留较强的更新;在纯背景区域,残差主要由噪声和纹理引起,表现为高频、随机或弱结构性扰动,NCEM 则更可能输出接近 0 的权重,从而抑制更新。这一过程与鲁棒加权最小二乘中“根据残差大小调整测量权重”的思想在形式和物理意义上是一致的。

另一方面,正则梯度 $\nabla S(T^{k-1})$ 的解析形式通常较难直接获取,因此 TEM 采用一个浅层卷积分支对其进行近似。具体做法是,将上一阶段目标估计与当前背景抠除后的残差进行通道拼接,形成
\[
X^{k}_{\mathrm{tar}} = [\,T^{k-1},\; D^{k-1}-B^{k}\,],
\]
然后通过两层 $3\times3$ 卷积与非线性激活提取目标相关先验,得到
\[
\Delta T^{k} = \mathrm{Conv}_{3\times3}\big(\sigma(\mathrm{Conv}_{3\times3}(X^{k}_{\mathrm{tar}}))\big).
\]
其中中间层采用固定通道数,最后一层将通道压回 1,使 $\Delta T^{k}$ 与 $T^{k-1}$ 在尺寸和通道上完全对齐,在功能上等价于对 $\nabla S(T^{k-1})$ 的一个数据驱动近似:网络可以在局部空间邻域内对目标边缘和细节结构进行增强,同时抑制孤立噪声与伪纹理,实现对正则先验的自适应建模。

将该近似代入理论更新式,TEM 中实际采用的更新形式为
\[
T^{k} = T^{k-1} + C^{k} \odot \Big(\alpha^{(k)}(D^{k-1}-B^{k}) - \beta^{(k)} \Delta T^{k}\Big).
\]
其中 $\alpha^{(k)}$ 控制数据项对更新方向的贡献,决定在多大程度上依赖当前残差 $(D^{k-1}-B^{k})$;$\beta^{(k)}$ 控制正则分支的步长大小,避免无约束地放大由 $\Delta T^{k}$ 产生的更新,从而在抑制噪声与保留目标之间取得平衡。由于 $\alpha^{(k)}$ 和 $\beta^{(k)}$ 作为阶段级标量参数由网络端到端学习得到,并与像素级权重 $C^{k}$ 共同作用,因此 TEM 在整体上既保留了第 3 章中带权梯度步的数学结构,又通过卷积网络与门控机制显式实现了对残差分布和目标结构的建模,使得在复杂背景和低信噪比条件下的小目标估计更加稳定、鲁棒。







\subsection{图像重建模块 IRM}


图像重建模块 \( \mathcal{M}^{(k)} \) 对应于 (3-21)。为了保持结构简洁并严格遵循物理模型 \( D = B + T + N \),
本文采用**无参数的线性重构形式**:
\[
D^k = \mathcal{M}^{(k)}(B^k + T^k) = B^k + T^k. \tag{4-7}
\]

在该设计下,每一阶段结束时都显式满足
\[
D^k \approx B^k + T^k,
\]
噪声与建模误差被保留在残差
\[
R^k = D^k - B^k - T^k
\]
中供下一阶段使用。若在实际应用中发现需要更强的重建能力,也可在 \( B^k + T^k \) 上附加一层轻量 \( 3 \times 3 \) 
卷积进行微调,但本文优先采用无参数形式以强调可解释性。



\subsection{小结 }

本章在第3章优化模型与迭代形式的基础上,构建了一种基于深度展开的红外小目标检测网络。其核心特点如下:

1. **结构上严格对应迭代过程**
网络由 \( K \) 个阶段串联,每一阶段都执行一次“背景估计 - 噪声估计 - 目标更新 - 图像重建”的流程,与 (3-22) 中的迭代步骤一一对应,保证了良好的物理和优化可解释性。


2. **背景模块 BEM 的浅层近端近似**
BEM 用两层卷积 + 残差结构近似背景近端算子,既保留了背景平滑/低维先验,又避免了显式 SVD 等高复杂度操作。


3. **噪声 - 杂波模块 NCEM 的显式建模**
NCEM 将重建残差映射为像素级置信度图 \( C^k \),为目标估计提供噪声相关的先验信息,使网络能在空间上自适应不同噪声水平和杂波强度。


4. **噪声感知的目标模块 TEM**
TEM 在构造正则梯度近似 \( \Delta T^k \) 时显式引入 \( C^k \),并通过可学习标量 \( \alpha^{(k)}, \beta^{(k)} \) 平衡数据项与正则项的贡献,实现了一种简单而有效的噪声感知目标更新机制,同时避免了直接对更新量进行矩阵乘法门控带来的不稳定性。


5. **简洁的图像重建模块 IRM**
IRM 采用 \( D^k = B^k + T^k \) 的线性重构形式,保证每一阶段都满足物理分解关系,将噪声和建模误差留在残差中供后续阶段使用。


这一网络结构充分融合了红外小目标的成像机理、优化建模以及噪声统计特性,为后续章节的损失函数设计与实验验证提供了清晰、严密的理论与结构基础。



\section{实验设计}

\subsection{实验设置}

\subsubsection{数据集}

\subsubsection{评估指标}

\subsubsection{实现细节}


\subsection{消融实验}

\subsubsection{实现细节}



\subsection{与现有最先进方法的比较}


\subsection{关于失败案例和模型先验的讨论}


\subsection{可视化分析}





\section{结果呈现,相对第三章的改进要突出}

